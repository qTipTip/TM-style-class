\documentclass{TMarticle}
\usepackage[]{lipsum} 
\definecolor{TMcodeBackground}{RGB}{240, 240, 240}
\definecolor{TMbulletinBackground}{RGB}{240, 240, 240}
\author{Ivar Stangeby}
\title{The TMarticle document class}

\begin{document}

\maketitle
\section{Code Listings}

\begin{TMcode}{C++}{testcode.cpp}{A test C++ program}
void main(int argc) {
    // a test function with comment
    std::cout << "a string!" << std::endl;
    return 0;
}
\end{TMcode}

\section{Tables and Figures}
\begin{TMtable}{X X X X}{ht!}{
        Presented is the computed integral, the absolute error in calculations
        as well as time elapsed for N integration steps. The time complexity of
        the integral itself is again $\mathcal{O}(N^6)$ however the numerical method is
        converging properly as opposed to the Legendre quadrature.
} 
        $N$ & Result & Absolute error & Time [sec]\\
        5& 0.1734& 0.0193& 0.0011\\ 
        10& 0.1864& 0.0063& 0.0675\\ 
        15& 0.1897& 0.0030& 0.8190\\
        20& 0.1910& 0.0016& 4.3892\\ 
\end{TMtable}
\section{Warnings and bulletins}

\begin{TMbulletin}{warning}{Test Warning}
    Malesuada ligula sociosqu faucibus a venenatis ridiculus ante scelerisque
    dui nulla leo platea condimentum vestibulum a aliquam. Libero litora
    ullamcorper justo diam nascetur parturient enim ad enim a nullam elit metus
    himenaeos dictum hac semper at adipiscing ac tempor laoreet hac parturient
    elementum.
\end{TMbulletin}
\begin{TMbulletin}{normal}{Test Normal}
    Parturient metus senectus ut dis ante sit a id dis urna imperdiet neque
    fermentum vehicula consectetur varius feugiat tempus himenaeos ad nisi
    curabitur.Ultricies dis parturient nulla vel vestibulum sodales fames
    faucibus quis.
\end{TMbulletin}
\begin{TMbulletin}{critical}{Test Critical}
    Iaculis ad ac vivamus scelerisque a ultrices a volutpat eget porta non mus
    scelerisque convallis dictumst.Condimentum velit consequat fringilla.
\end{TMbulletin}

\section{Proclamations}

\begin{theorem}[Euclid]
   This is a test proclamation with a lot of mathematics like $x^2 = 7$ and $i^2 = -1$.
   \[
       f(x) = x \int_2^7 g(x) \, dx.
   \]
\end{theorem}
\begin{proof}
   It is easy to show the above since it follows from already proven results.
\end{proof}
\begin{lemma}[TestLemma]
    Please ignore.
\end{lemma}
\begin{proposition}[Euclid]
   This is a test proclamation with a lot of mathematics like $x^2 = 7$ and $i^2 = -1$.
   \[
       f(x) = x \int_2^7 g(x) \, dx.
   \]
\end{proposition}
\begin{corollary}[Euclid 2]
    Please ignore.
\end{corollary}
\section{Colors}
\end{document}
