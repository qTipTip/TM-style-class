\documentclass{TMarticle}
\usepackage[]{lipsum} 
\usepackage[]{paralist} 
\usepackage[]{amsmath} 
\usepackage[]{amssymb} 
\definecolor{TMcodeBackground}{RGB}{240, 240, 240}
\definecolor{TMbulletinBackground}{RGB}{240, 240, 240}
\author{Ivar Stangeby}
\title{The TMarticle document class}

\begin{document}

\maketitle
\section{Code Listings}

\begin{TMcode}{C++}{testcode.cpp}{A test C++ program}
void main(int argc) {
    // a test function with comment
    std::cout << "a string!" << std::endl;
    return 0;
}
\end{TMcode}

\section{Tables and Figures}
\begin{TMtable}{X X X X}{ht!}{
        Presented is the computed integral, the absolute error in calculations
        as well as time elapsed for N integration steps. The time complexity of
        the integral itself is again $\mathcal{O}(N^6)$ however the numerical method is
        converging properly as opposed to the Legendre quadrature.
} 
        $N$ & Result & Absolute error & Time [sec]\\
        5& 0.1734& 0.0193& 0.0011\\ 
        10& 0.1864& 0.0063& 0.0675\\ 
        15& 0.1897& 0.0030& 0.8190\\
        20& 0.1910& 0.0016& 4.3892\\ 
\end{TMtable}
\section{Warnings and bulletins}

\begin{TMbulletin}{warning}{Test Warning}
    Malesuada ligula sociosqu faucibus a venenatis ridiculus ante scelerisque
    dui nulla leo platea condimentum vestibulum a aliquam. Libero litora
    ullamcorper justo diam nascetur parturient enim ad enim a nullam elit metus
    himenaeos dictum hac semper at adipiscing ac tempor laoreet hac parturient
    elementum.
\end{TMbulletin}
\begin{TMbulletin}{normal}{Test Normal}
    Parturient metus senectus ut dis ante sit a id dis urna imperdiet neque
    fermentum vehicula consectetur varius feugiat tempus himenaeos ad nisi
    curabitur.Ultricies dis parturient nulla vel vestibulum sodales fames
    faucibus quis.
\end{TMbulletin}
\begin{TMbulletin}{critical}{Test Critical}
    Iaculis ad ac vivamus scelerisque a ultrices a volutpat eget porta non mus
    scelerisque convallis dictumst.Condimentum velit consequat fringilla.
\end{TMbulletin}

\section{Proclamations}

\begin{theorem}[Fredholm Alternative]
    For any fixed $\mu \in \mathbb{C}$ the Fredholm alternative holds for the
    second kind Fredholm integral equation
    \begin{equation}
        \notag
        (I - \mu K) u = f.
    \end{equation}
    That is, either
    \begin{inparaenum}[a)]
    \item $\mu$ is not a characteristic value of $K$ and the equation has a
        unique solution $u \in \mathcal{H}$ for any given $f \in \mathcal{H}$;
        or \item $\mu$ is a characteristic value of $K$ and the corresponding
        homogeneous equation has non-zero solutions, while the inhomogeneous
        equation has (non-unique) solutions if and only if $f$ is orthogonal to
        the subspace $\mathrm{Ker}(I - \bar{\mu}K^*)$. 
    \end{inparaenum}
\end{theorem}
\begin{proof}
    Left as an exercise for the reader.
\end{proof}
\begin{lemma}[Volterra integral operator]
    If $K$ is a Volterra integral operator on $\mathcal{H}$ then there exists a
    constant $C > 0$ such that $\|K^n\|_{\mathcal{H}} \leq C^n / n!$, for any
    integer $n \geq 1$.
\end{lemma}

\section{Icons}
\label{sec:icons}

The icons are taken from the link given in the readme-file. Slight colorations
done by me.\\

    \includegraphics[width=0.17\linewidth]{images/code2.png}
    \includegraphics[width=0.17\linewidth]{images/critical.png}
    \includegraphics[width=0.17\linewidth]{images/warning.png}
    \includegraphics[width=0.17\linewidth]{images/normal.png}
    \includegraphics[width=0.17\linewidth]{images/stats.png}

\end{document}
