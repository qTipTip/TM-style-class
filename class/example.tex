\documentclass{TMarticle}
\usepackage[]{lipsum} 
\definecolor{TMcodeBackground}{RGB}{240, 240, 240}
\definecolor{TMbulletinBackground}{RGB}{240, 240, 240}
\author{Ivar Stangeby}
\title{The TMarticle document class}

\begin{document}

\maketitle
\section{Code Listings}

\begin{TMcode}{C++}{testcode.cpp}{A test C++ program}
void main(int argc) {
    // a test function with comment
    std::cout << "a string!" << std::endl;
    return 0;
}
\end{TMcode}

\section{Tables and Figures}
\begin{TMtable}{X X X X}{ht!}{
        This is a caption!
    }
    $N$ & Result & Absolute error & Time [sec]\\
    5& 0.1734& 0.0193& 0.0011\\
    10& 0.1864& 0.0063& 0.0675\\
    15& 0.1897& 0.0030& 0.8190\\
\end{TMtable}
\section{Warnings and bulletins}

\begin{TMbulletin}{warning}{Test Warning}
    This is a test, it has nothing to do with anything what-so-ever. I am only
    typing to you here in a stream of conciousness type of style to fill
    something in. Since you are still reading this, have you read Virginia
    Woolf? 
\end{TMbulletin}
\begin{TMbulletin}{normal}{Test Normal}
\end{TMbulletin}
\begin{TMbulletin}{critical}{Test Critical}
\end{TMbulletin}

\section{Proclamations}

\begin{theorem}[Euclid]
   This is a test proclamation with a lot of mathematics like $x^2 = 7$ and $i^2 = -1$.
   \[
       f(x) = x \int_2^7 g(x) \, dx.
   \]
\end{theorem}
\begin{proof}
   It is easy to show the above since it follows from already proven results.
\end{proof}
\begin{lemma}[TestLemma]
    Please ignore.
\end{lemma}
\begin{proposition}[Euclid]
   This is a test proclamation with a lot of mathematics like $x^2 = 7$ and $i^2 = -1$.
   \[
       f(x) = x \int_2^7 g(x) \, dx.
   \]
\end{proposition}
\begin{corollary}[Euclid 2]
    Please ignore.
\end{corollary}
\section{Colors}
\end{document}
